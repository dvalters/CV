% resume.tex
% vim:set ft=tex spell:

\documentclass[10.5pt,a4]{article}
\usepackage[letterpaper,margin=2cm]{geometry}
\usepackage[utf8]{inputenc}
\usepackage{mdwlist}
\usepackage[T1]{fontenc}
\usepackage{textcomp}
\usepackage{tgpagella}
\pagestyle{empty}
\setlength{\tabcolsep}{0em}

%\renewcommand{\familydefault}{\sfdefault}
%\usepackage{helvet}
%\renewcommand{\familydefault}{\sfdefault}

% indentsection style, used for sections that aren't already in lists
% that need indentation to the level of all text in the document
\newenvironment{indentsection}[1]%
{\begin{list}{}%
	{\setlength{\leftmargin}{#1}}%
	\item[]%
}
{\end{list}}

% opposite of above; bump a section back toward the left margin
\newenvironment{unindentsection}[1]%
{\begin{list}{}%
	{\setlength{\leftmargin}{-0.5#1}}%
	\item[]%
}
{\end{list}}

% format two pieces of text, one left aligned and one right aligned
\newcommand{\headerrow}[2]
{\begin{tabular*}{\linewidth}{l@{\extracolsep{\fill}}r}
	#1 &
	#2 \\
\end{tabular*}}

% make "C++" look pretty when used in text by touching up the plus signs
\newcommand{\CPP}
{C\nolinebreak[4]\hspace{-.05em}\raisebox{.22ex}{\footnotesize\bf ++}}

% and the actual content starts here
\begin{document}

\begin{center}
{\LARGE \textbf{Declan A. Valters}} \\
\parskip=0.1em
{Scientific Software Engineer -- Met Office} \\
\parskip=0.4em
%14 Otley Mount\ \textbullet
%\ \ East Morton\ \textbullet \ Keighley  \\
%\ BD20 5TD\ \textbullet \ United Kingdom \ \textbullet \
dvalts.io  \ \textbullet
\ \ declan.valters@gmail.com \ \textbullet \ github.com/dvalters
\end{center}


\hrule
\vspace{-0.4em}
\subsection*{Personal Statement}
I am a scientific software engineer interested in atmospheric, hydrological, and geomorphological modelling, as well as topographic data analysis. I am interested particularly in the links between weather and land surface processes and the development of numerical models that underpin research in land surface dynamics using high-performance computing.
\vspace{1em}

\hrule
\vspace{-0.4em}
\subsection*{Education}
\begin{itemize}
	\parskip=0.1em
	
	\item 
	\headerrow
		{\textbf{PhD in Earth, Atmospheric, and Environmental Science}}
		{\textbf{University of Manchester}}
	\\
	\headerrow
		{\emph{Thesis}: Modelling catchment sensitivity to rainfall resolution and}
		{\emph{September 2013 -- March 2017}}
	\headerrow
		{\hspace{10mm} erosional parameterisation in simulations of flash floods in the UK}
		{}
  \headerrow
    {\emph{Supervisors:} Prof David Schultz, Dr Simon Brocklehurst}
  
	\item 
	\headerrow
		{\textbf{Master in Earth Science (Hons., 1\textsuperscript{st} Class)} }
		{\textbf{University of Edinburgh}}
	\\
	\headerrow
		{\emph{Thesis}: Extracting tectonic information using statistical methods of river profile analysis}
		{\emph{2009 -- 2013}}
	\headerrow
    {\emph{Supervisor:} Prof Simon Mudd} 
    {}
\end{itemize}

\hrule
\vspace{-0.4em}
\subsection*{Experience and Software Projects}
\begin{itemize}
	\parskip=0.1em
	\item
	\headerrow
		{\textbf{Met Office -- Weather Science IT}}
		{\textbf{metoffice.gov.uk}}
	\\
	\headerrow
		{\emph{Scientific Software Engineer}}
		{\emph{March 2017 -- Present}}
	\begin{itemize*}
		\item Development of the Cylc software package, a scientific workflow manager and scheduler. 
		\item Development of the Rose software framework for configuration of meteorological applications.
			\end{itemize*}

	\item
	\headerrow
		{\textbf{HAIL-CAESAR: A numerical landscape evolution model for HPC}}
		{\textbf{dvalts.io/HAIL-CAESAR}}
	\\
	\headerrow
		{\emph{PhD software project}}
		{\emph{September 2013 -- 2017}}
	\begin{itemize*}
		\item A C++ cellular automaton model ported to HPC (High performance computing) facilities through a shared-memory parallelism model (OpenMP). 
		\item I translated and developed the CAESAR-Lisflood numerical model from a C{\#}/.NET application into a platform-independent code suitable for high-performance computer use such as ensemble simulations and sensitivity analyses.
	\end{itemize*}

	\item
	\headerrow
		{\textbf{Land Surface Dynamics Topographic Toolbox}}
		{\textbf{lsdtopotools.github.io}}
	\\
	\headerrow
		{\emph{Open source developer/contributor}}
		{\emph{2012 -- Present}}
	\begin{itemize*}
		\item Object-oriented {\CPP} topographic analysis and modelling package developed with the Land Surface Dynamics research group at Edinburgh. The continuing aim of the project is to implement state-of-the art algorithms as they are published in academic literature. A key aim of LSDTopoTools is to facilitate reproducible scientific data analysis for large topographic datasets.
		\item My specific role was to develop the statistical analysis tools (\CPP), visualisation (Python), and automation scripts (Python) for task-farming sensitivity analyses.
	\end{itemize*}
	
	\item
	\headerrow
		{\textbf{Met Office -- Satellite Applications}}
		{\textbf{nwpsaf.eu}}
	\\
	\headerrow
		{\emph{Full-stack web developer}}
		{\emph{July - October, 2015}}
	\begin{itemize*}
		\item Redevelopment of the Met Office/European Meteorological Satellite facility website. A public website used for the retrieval of post-processed satellite data and imagery.
		\item Designed and implemented a MySQL database for satellite image metadata, integrated with a Javascript front-end for retrieval and rendering of data and imagery.
		\item I wrote several tools for keeping the database maintained automatically (Shell scripts/Python/PHP) as new data were added. 
			\end{itemize*}

\end{itemize}

\hrule
\vspace{-0.4em}
\subsection*{Publications}
\begin{itemize}

	\parskip=0.1em
	\item	
	\headerrow
	{\textbf{In preparation}}
	
	\begin{itemize*}
			\item \textbf{Valters, D.A.,} et al. (in prep.) \textit{HAIL-CAESAR: A cellular automaton hydrodynamic landscape evolution model parallelised for shared-memory computing architectures.} Geoscientific Model Development	
	\parskip=0.5em 	
				\item \textbf{Valters, D.A.,} et al. (in prep.) \textit{Sensitivity of a flood-inundation model to rainfall distribution and erosional parameterisation.} Hydrology and Earth System Sciences.
	\end{itemize*}	
	
	\item	
	\headerrow
	{\textbf{2017}}
	
	\begin{itemize*}

   \parskip=0.5em 
    \item Clubb, F.J. , Mudd, S.M., Milodowski, D.T., \textbf{Valters, D.A.}, Slater, L.J., Hurst, M.D., and Limaye, A.B (2017) Geomorphometric delineation of floodplains and terraces from objectively defined topographic thresholds, Earth Surf. Dynam.
	\parskip=0.5em
	
  \end{itemize*}
  
  \item	
	\headerrow
	{\textbf{2016}}
	
	\begin{itemize*}
		\item \textbf{Valters, D.A.} (2016). \textit{Modelling Geomorphic Systems: Landscape Evolution.} In: Cook, S.J., Clarke, L.E. \& Nield, J.M. (Eds.) Geomorphological Techniques (Online Edition). British Society for Geomorphology; London, UK. ISSN: 2047-0371.
	\end{itemize*}
	
	
	\item	
	\headerrow
		{\textbf{2014}}
		
	\begin{itemize*}
		\item Mudd, S.M., Attal, M., Milodowski, D.T., Grieve, S.W.D. and \textbf{Valters, D.A.} (2014). \textit{A statistical framework to quantify spatial variation in channel gradients using the integral method of channel profile analysis}, Journal of Geophysical Research: Earth Surface
	\end{itemize*}
\end{itemize}

\hrule
\vspace{-0.4em}
\subsection*{Conference Presentations and Abstracts}
\begin{itemize}
	\parskip=0.1em

	\item
	\headerrow
	{\textbf{2016}}
		
	\begin{itemize*}
	\item \textbf{Valters, D.A.} (2016) \textit{Frontiers in geomorphological computing.} 1st annual Research Software Engineers conference, Manchester, UK.
	\parskip=0.5em
	\item \textbf{Valters, D.A.}, \& Brocklehurst, S. H. (2016) \textit{Topographic signatures of spatially-limited storm morphologies revealed from numerical landscape evolution modelling.} Geophysical Research Abstracts, EGU General Assembly, Abstract 18-14328.
	\parskip=0.5em
	\item Clubb, F.J., Mudd, S.M., Milodowski, D.T., and \textbf{Valters, D.A.} (2016) Geomorphometric delineation of floodplains
and terraces from slope and channel relief thresholds, AGU Fall Meeting, Poster.
	\end{itemize*}
	
	\item	
	\headerrow
	{\textbf{2015}}	
	
	\begin{itemize*}	
  \item \textbf{Valters, D.A.}, Brocklehurst, S. and Schultz, D., (2015). Sensitivity of hydro-geomorphic processes to catchment-scale variations in rainfall distribution. In EGU General Assembly Conference Abstracts (Vol. 17).	
	\parskip=0.5em
  \item	Mudd, S.M., Grieve, S.W.D., Milodowski, D.T., Hurst, M.D., Clubb, F.J.,   \textbf{Valters, D.A.}, (2015) \textit{LSDTopoToolBox: Open source geomorphology.} Presented at the BSG Annual General Meeting, Southampton.
  \end{itemize*}
  
	\item
	\headerrow
	{\textbf{2014}}

	\begin{itemize*}
		\item \textbf{Valters, D.A.} (2014). \textit{Modelling landscape sensitivity to stormier climates}, British Society for Geomorphology Annual Conference, BSG Annual General Meeting, Manchester	
	\parskip=0.5em
		\item \textbf{Valters, D.A.} and Mudd, S.M. (2014). \textit{Extracting tectonic information using the integral method of river profile analysis: applications along the Wasatch fault, Utah}, Geophysical Research Abstracts, EGU General Assembly, Abstract EGU2014-16074-1.
	\parskip=0.5em	
	\item Mudd, S.M., Attal, M., Milodowski, D.T., Grieve, S.W.D. and \textbf{Valters, D.A.} (2014). \textit{A statistical technique for identifying channels of different steepness in transient landscapes}, Geophysical Research Abstracts, EGU General Assembly, Abstract EGU2014-15780.
	\end{itemize*}

\end{itemize}


\hrule
\vspace{-0.4em}
\subsection*{Technical Skills}
	\parskip=0.1em
	\paragraph*{Programming Languages \& Software}
		\begin{itemize*}
		\item My current working languages are {\textbf{\CPP}} and \textbf{Python} (including NumPy, Matplotlib). I've experience in writing object-oriented and procedural style code. 
		\item Experience in HPC applications including implementing \textbf{OpenMP}-style parallelism, as well as \textbf{MPI} approaches to parallelisation. 
		\item Experience in using \textbf{subversion} and \textbf{git} version control systems. 
		\item Previously I've worked on projects using \textbf{Javascript} and \textbf{PHP} for web development. 
		\item Basic knowledge of \textbf{Fortran}, \textbf{Matlab}, \textbf{C}, and shell scripting in Linux. 
	\item  \textbf{ArcGIS} 9 \& 10, \textbf{GRASS-GIS} and \textbf{QGIS}. 
	\item Experience in using and modifying the \textbf{WRF} numerical weather prediction model and familiarity with the Met Office \textbf{Unified Model} (UM).
	\end{itemize*}


\hrule
\vspace{-0.4em}
\subsection*{Professional Development}
\begin{itemize}
	\item 
	\headerrow
		{\textbf{Programming/Technical courses}}
		{\textbf{2-3 day courses, provided by ARCHER/EPCC}}
	\\
		\headerrow
		{Fortran Modernisation}
		{\emph{February 2017}}
	\\
	\headerrow
		{Writing scalable parallel applications with MPI}
		{\emph{December 2016}}
	\\
	\headerrow
		{Advanced MPI}
		{\emph{September 2016}}
		\\
	\headerrow
		{Advanced OpenMP}
		{\emph{August 2016}}
	\\
	\headerrow
		{Message-passing programming with MPI}
		{\emph{July 2016}}
	\\
	\headerrow
		{Single-node performance optimisation}
		{\emph{December 2015}}
	\\
	\headerrow
		{Shared Memory programming with OpenMP}
		{\emph{December 2015}}
	\\
	\headerrow
		{Extended introduction to CUDA}
		{\emph{November 2015}}
\end{itemize}

\begin{itemize}
	\item 
	\headerrow
		{\textbf{Numerical Weather Prediction Model training}}
		{}
	\\
		\headerrow
		{The Weather Research and Forecasting Model (WRF)} 
		{\textit{NCAS/NCAR -- October 2013}}
  \\
		\headerrow
		{Met Office Unified Model (UM)}
		{\textit{NCAS/University of Reading -- December 2014}}
\end{itemize}

\begin{itemize}
	\item 
	\headerrow
		{\textbf{Scientific training}}
		{}
	\\
		\headerrow
		{NERC/JBA Extreme Flood Forecasting and Management}
		{\emph{5 days -- January 2015}}
	\\
		\headerrow
		{NCAS Atmospheric Science Summer School}
		{\emph{2 weeks -- September 2013}}

\end{itemize}

\begin{itemize*}
	\item 
	\headerrow
		{\textbf{Professional memberships}}
		{}
	\headerrow {British Society for Geomorphology}{}
	\headerrow {UK Research Software Engineers Network}{}

\end{itemize*}

\hrule
\vspace{-0.4em}
\subsection*{Other Roles and Service} 
	\parskip=0.1em
\begin{itemize*}
	\item
	\headerrow
		 {Journal of Open Source Software -- reviewer}
		 {\textit{2016 -- Present}}
	\item 
	\headerrow 
	  {British Society for Geomorphology -- Web Officer}
	  {\textit{2015 -- 2017}}
	\item
	\headerrow
		 {Teaching Assistant -- University of Manchester}
		 {\textit{2013 -- 2016}}
  \headerrow
      {Courses taught:}
      {}
		\begin{itemize*}
		\item Fortran and Matlab for engineers
    \item Earth Surface Processes (Geomorphology)
    \item Engineering Geology
    \item Earth Resources
    \item Our Earth (Open Online Course Moderation)
		\item Global tectonics
			\end{itemize*}
\end{itemize*}

\hrule
\vspace{-0.4em}
\subsection*{Grants and Awards}
	\parskip=0.1em
\begin{itemize*}

	\item 
	\headerrow 
	  {5th Intel Xeon Phi Access Programme}
	  {STFC, Hartree Centre -- 4 months trial}

	\item
	\headerrow
		 {Mackay Greenland Scholarship}
		 {UoE Award for Greenland-based research project -- £1000}
	\item Undergraduate Class Medal


\end{itemize*}


%\hrule
%\vspace{-0.4em}
%\subsection*{Referees}
%
%
%
%\begin{indentsection}{\parindent}
%\hyphenpenalty=1000
%
%Available on request.
%\end{indentsection}

\end{document}
